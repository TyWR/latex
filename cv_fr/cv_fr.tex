% -- Encoding UTF-8 without BOM
% -- XeLaTeX => PDF (BIBER)
\documentclass[]{cv-style} 
%
\sethyphenation[variant=british]{french}{} 
\usepackage[utf8]{inputenc}
%
\begin{document}
\header{Tanguy}{ VIVIER}
%------------------------------------------------------------------------------
%					   		ASIDE
%------------------------------------------------------------------------------
\begin{aside}
%
\section{CONTACT} \vspace{0.1cm}
Tél: +33 6 33 05 90 61
tanguy.viv@gmail.com
%
\section{ADRESSE} \vspace{0.1cm}
136 Chemin de Rosset,
Saint-Jeoire Prieuré,
73190,
France
%
\section{INFO} \vspace{0.1cm}
22 ans
Permis B
Secouriste du travail
%
\section{LIENS} \vspace{0.1cm}
\href{https://github.com/TyWR}{\textbf{github.com/TyWR}}
\href{https://www.linkedin.com/in/tanguy-vivier/}{\textbf{in/tanguy-vivier}}
\href{https://tywr.github.io/}{\textbf{tywr.github.io}}
%
\section{LANGAGES} \vspace{0.1cm}
Français (Maternelle)
Anglais (C1)
%
\section{OUTILS} \vspace{0.1cm}
Linux/MacOS
NeoVim
ZSH
Jupyter
\LaTeX{}
%
\section{PASSIONS} \vspace{0.1cm}
Musique
Cuisine
Voyages
Natation
%
\end{aside}
%------------------------------------------------------------------------------
%					   EDUCATION SECTION
%------------------------------------------------------------------------------
\section{COMPÉTENCES TECHNIQUES}
\vspace{-0.3cm}
%
Ingénieur civil des Mines avec de larges connaissances de fond en mathématiques 
appliquées, en data-science (Machine Learning, Statistiques, Traitement du
signal etc.) et en informatique, avec une expérience de développement sur des
projets scientifiques divers. \\[0.1cm]
%------------------------------------------------------------------------------
\textbf{DATA-SCIENCE} \\[0.1cm]
\begin{entrylist}
%
\entry
{Maîtrise}
{{\normalfont Python (Matplotlib, SkLearn, Numpy, Pandas, Flask)}}
{}{\vspace{-0.5cm}}
%
\entry
{Familier}
{{\normalfont Matlab, R, d3.js, PyTorch}}
{}{\vspace{-0.5cm}}
%
\entry
{Notions}
{{\normalfont Tensorflow, Spark, Hadoop, AWS S3, Redshift}}{}{}
%
\end{entrylist}
\vspace{-0.5cm}\\
%------------------------------------------------------------------------------
\textbf{SOFTWARE ENGINEERING} \\[0.1cm]
\begin{entrylist}
%
\entry
{Maîtrise}
{{\normalfont git, Unix}}
{}{\vspace{-0.5cm}}
%
\entry
{Familier}
{{\normalfont SQL, Docker, C/C++, bash, JavaScript, HTML/CSS, Parallel 
Processing}}
{}{\vspace{-0.5cm}}
%
\entry
{Notions}
{{\normalfont Haskell, Scala}}{}{\vspace{-0.5cm}}
%
\end{entrylist}
%------------------------------------------------------------------------------
%					       FORMATION
%------------------------------------------------------------------------------
\section{FORMATION}
\begin{entrylist}
%------------------------------------------------------------------------------
\entry
{2016-2019}
{INGENIEUR CIVIL DES MINES\\
{\normalfont 
    Département Génie Industriel et Mathématiques Appliquées \\
    Data Science (Machine Learning, Analyse de données, Optimisation, 
    Statistique...), 
    Informatique (Python, C/C++ ...) \\
}}
{\vspace{-0.4cm}}
{\addfontfeature{Color=lightgray} \small Ecole des Mines de Nancy}
%------------------------------------------------------------------------------
\entry
{2018}
{ECHANGE ERASMUS \\
{\normalfont 
    Data Science et Informatique (Modèles de régression, Modélisation 
    Mathématique, Calcul Parallèle...) \\
}}
{\vspace{-0.4cm}}
{\addfontfeature{Color=lightgray} \small NTNU, Trondheim, Norvège}
%------------------------------------------------------------------------------
\entry
{2014-2016}
{CLASSES PREPARATOIRES AUX GRANDES ECOLES \\
{\normalfont 
    MPSI / MP* (Mathématiques, Physique, Informatiques)\\
}}
{\vspace{-0.4cm}}
{\addfontfeature{Color=lightgray} \small Lycée Berthollet, Annecy}
%
\end{entrylist}
%------------------------------------------------------------------------------
%					  WORK EXPERIENCE SECTION
%------------------------------------------------------------------------------
\section{EXPERIENCES}
\begin{entrylist}
%
\entry
{Fév. 2019}
{DEVELOPPEUR PYTHON / DATA SCIENTIST (6 mois)}
{Genève, Suisse}
{\jobtitle{Fondation Campus Biotech Geneva}\\
    $\circ$ Prise en charge du développement d'outils avec Python pour 
    l'analyse et la visualisation de données neurophysiologiques: Analyse 
    spectrale et temps-fréquence, ICA.\\
    $\circ$ Développement d'une plateforme pour le traitement à grande échelle 
    de workflows pour le traitement de données neurophysiologiques, utilisant 
    du calcul parallèle et des outils de déploiement (Docker). \\
\textbf{Utilisé:} git, Python, Docker.
\vspace{0.2cm}
}
%------------------------------------------------------------------------------
\entry
{Juin 2018}
{STAGE DE RECHERCHE (2 mois)}
{Kyushu, Iizuka, Japon}
{\jobtitle{Kyutech}\\
    Stage de recherche dans le domaine de la physique du spin. Développement 
    de solutions pour le traitement de données, et réalisation de 
    manipulations. 
\vspace{0.2cm}
}
 %-----------------------------------------------------------------------------
\entry
{2017-2018}
{JUNIOR-ENTREPRISE}
{Nancy, France}
{\jobtitle{Mines Services}\\
    Auditeur interne et suiveur d'étude dans la junior-entreprise des Mines 
    de Nancy.
\vspace{0.2cm}
}
%
\end{entrylist}
%------------------------------------------------------------------------------
%					  WORK EXPERIENCE SECTION
%------------------------------------------------------------------------------
\newpage
\hphantom\\
%
\section{PORTFOLIO}
\textbf{OUTIL DE VISUALISATION, NEUROSCIENCES ($\sim$3mois)}\\
\url{https://github.com/fcbg-hnp/mnelab}\\
    Prise en charge du développement d'un outil open-source pour la 
    visualisation et le traitement de données neurophysiologiques. Le but étant
    de présenter un logiciel complet permettant de nettoyer les données, et 
    d'appliquer des analyses complexes. (Analyse spectrale, séparation de 
    source, machine-learning etc.). Le logiciel est désormais utilisé par les 
    chercheurs sur le campus Biotech pour l'exploration des données.\\
\textbf{Utilisé:} Python (PyQT, Matplotlib, Numpy, MNE), traitement du signal 
\\[0.4cm]
%------------------------------------------------------------------------------
\textbf{OPTIMISATION D'UNE PIECE POUR LE CHAUFFAGE ($\sim$10mois)}\\
    Projet d'un an visant à optimiser la forme d'une pièce afin de maximiser 
    la température moyenne à l'intérieur, pour une pièce de taille fixée.\\
    $\circ$ Le problème a été modélisé en utilisant des éléments finis avec 
    maillage adaptatif et une équation de la chaleur\\
    $\circ$ Utilisation de différentes techniques pour l'optimisation 
    (Déplacements élémentaires des coins, approximation de la fonction par un 
    réseau de neurones etc.) \\
\textbf{Utilisé:} Python, Matlab, Optimisation \\[0.4cm]
%------------------------------------------------------------------------------
\textbf{PRÉVISIONS SPORTIVES}\\
Modélisation d'un tournoi sportif à l'aide de machine-learning (Régression 
linéaire de type Poisson). Simulation des issus possibles à l'aide d'une 
méthode Monte-Carlo, et rendu d'un tableau avec les probabilités pour chaque 
équipe de gagner. Le projet comprenant une phase de transformation des données,
et une phase de choix et d'élaboration d'un modèle.\\
\textbf{Utilisé:} R, Modèles linéaires, Méthode de Monte-Carlo \\[0.4cm]
%------------------------------------------------------------------------------
\textbf{PRÉVISION DES AURORES BOREALES, VISUALISATION}\\
\url{https://northern-lights.herokuapp.com/}\\
    Projet personnel visant à réaliser une interface interactive permettant de
    visualiser les prévisions d'aurores boréales sur une carte. Le site est 
    désormais utilisable.\\
\textbf{Utilisé:} Python (Flask), d3.js \\[0.4cm]
%------------------------------------------------------------------------------
%					  WORK EXPERIENCE SECTION
%------------------------------------------------------------------------------
\section{REFERENCES}
\textbf{Gwénaël BIROT} \\
    EEG/BCI Platform Manager, \\
    Fondation Campus Biotech Geneva,\\
\url{gwenael.birot@fcbg.ch}
%
\end{document}






