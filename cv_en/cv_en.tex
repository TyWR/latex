% -- Encoding UTF-8 without BOM
% -- XeLaTeX => PDF (BIBER)

\documentclass[]{cv-style}         
\sethyphenation[variant=british]{english}{} 

\begin{document}
%
\header{Tanguy}{ VIVIER}
%------------------------------------------------------------------------------
%						    ASIDE
%------------------------------------------------------------------------------
\begin{aside}
%
\section{CONTACT}\vspace{0.1cm}
Tel: +33 633 059061
tanguy.viv@gmail.com
%
\section{ADDRESS}\vspace{0.1cm}
136 Chemin de Rosset,
Saint-Jeoire Prieuré,
73190,
France
%
\section{INFO}\vspace{0.1cm}
French
22 years old
Driver License
First-Aider
%
\section{LINKS}\vspace{0.1cm}
\href{https://github.com/TyWR}{Github: TyWR}
\href{https://www.linkedin.com/in/tanguy-vivier/}{Linkedin}
\href{https://tywr.github.io/}{Website}
%
\section{LANGUAGES}\vspace{0.1cm}
French (native)
English (fluent)
%
\section{TOOLS}\vspace{0.1cm}
Linux/MacOS
NeoVim
ZSH
Jupyter
\LaTeX{}
%
\section{HOBBIES} \vspace{0.1cm}
Music
(Bass and Drums)
Swimming
Travelling
Cooking
%
\end{aside}
%------------------------------------------------------------------------------
%						 TECHNICAL SKILLS
%------------------------------------------------------------------------------
\vspace{0.15cm}
\section{TECHNICAL SKILLS}
\vspace{-0.25cm}
%
Strong knowledge in applied Mathematics (Machine-Learning, Statistics, Signal 
Processing, Optimization, Calculus...) and Computer Science. Experience in 
medium-scale software engineering projects involving data-science and signal 
processing skills. \\[0.15cm] 
%------------------------------------------------------------------------------
\textbf{DATA-SCIENCE:} \\[0.15cm]
\begin{entrylist}
%
\entry
{Proficient}
{{\normalfont Python (Matplotlib, SkLearn, Numpy, Pandas, Flask)}}
{}{\vspace{-0.5cm}}
%
\entry
{Familiar}
{{\normalfont Matlab, R, d3.js, PyTorch}}
{}{\vspace{-0.5cm}}
%
\entry
{Knowledge}
{{\normalfont Tensorflow, Spark, Hadoop, AWS S3, Redshift}}{}{}
%
\end{entrylist}
\vspace{-0.5cm}\\
%------------------------------------------------------------------------------
\textbf{SOFTWARE ENGINEERING:} \\[0.15cm]
\begin{entrylist}
%
\entry
{Proficient}
{{\normalfont git, Unix}}
{}{\vspace{-0.5cm}}
%
\entry
{Familiar}
{{\normalfont SQL, Docker, C/C++, bash, JavaScript, HTML/CSS, Parallel 
Processing}}
{}{\vspace{-0.5cm}}
%
\entry
{Knowledge}
{{\normalfont Haskell, Scala}}
{}{\vspace{-0.5cm}}
%
\end{entrylist}
%------------------------------------------------------------------------------
%					        CURRICULUM
%------------------------------------------------------------------------------
\section{CURRICULUM}
\begin{entrylist}
%------------------------------------------------
\entry
{2016-2019}
{INGENIEUR CIVIL DES MINES \\ MASTER OF SCIENCE AND EXECUTIVE ENGINEERING\\
{\normalfont 
    One of France's top Master’s level engineering schools. \\
    Multidisciplinary approach that harmoniously blends basic scientific and
    technical education, technological education, and a solid initiation to
    the economic, social and human realities of industry.\\
}}
{\vspace{-0.4cm}}
{\addfontfeature{Color=lightgray} \small Ecole Nationale Supérieure des Mines 
de Nancy, France}
%------------------------------------------------
\entry
{2018}
{ERASMUS EXCHANGE \\
{\normalfont 
    Semester abroad during my engineering master degree in France within the
    faculty of Information Technology and Electrical Engineering\\
}}
{\vspace{-0.4cm}}
{\addfontfeature{Color=lightgray} \small NTNU, Trondheim, Norway}
%------------------------------------------------
\entry
{2014-2016}
{CLASSES PREPARATOIRES AUX GRANDES ECOLES \\
{\normalfont 
    Preparatory class to take national competitive exams for admission to the
    highly selective French engineering schools "Grandes écoles". 
    Mathematics, Physics, IT.\\
}}
{\vspace{-0.4cm}}
{\addfontfeature{Color=lightgray} \small Lycée Berthollet, Annecy, France}
%
\end{entrylist}
%------------------------------------------------------------------------------
%						EXPERIENCES
%------------------------------------------------------------------------------
\section{EXPERIENCES}
\begin{entrylist}
%
\entry
{Feb. 2019}
{SOFTWARE ENGINEER / DATA SCIENTIST (6 months)}
{Geneva, Switzerland}
{\jobtitle{Fondation Campus Biotech Geneva}\\
    $\circ$ Developing user-friendly softwares for neurophysiological data
    analysis, with a focus on frequency and time-frequency analysis, source
    separation (PCA, ICA...).\\
    $\circ$ Developing large scale computing workflows using parallel 
    processing and their deployment tools (Docker).\\
\textbf{Utilized} git, Python for data analysis and user interface, Docker.
\vspace{0.2cm}
}
%------------------------------------------------------------------------------
\entry
{Jun. 2018}
{SHORT INTERNSHIP (2 months)}
{Kyushu, Iizuka, Japan}
{\jobtitle{Kyutech}\\
    Research internship on the subject of spintronics, at the FUKUMA
    Laboratory, by carrying out experiments and data analysis.
\vspace{0.2cm}
}
%------------------------------------------------------------------------------
\entry
{2017-2018}
{JUNIOR-ENTERPRISE CONSULTING}
{Nancy, France}
{\jobtitle{Mines Services}\\
    Internal auditor of a student association in charge of giving consulting
    work to clients by giving missions to students.\\
\vspace{0.2cm}
}
%
\end{entrylist}
%------------------------------------------------------------------------------
%						    PROJECTS
%------------------------------------------------------------------------------
\newpage
\hphantom\\
%
\section{PROJECTS}
\textbf{OPEN SOURCE VISUALIZATION TOOL FOR EEG DATA-ANALYSIS}\\
\url{https://github.com/fcbg-hnp/mnelab}\\
    $\circ$ Carrying out the development of an open-source graphical user 
    interface to visualize and process raw neurophysiological data. The goal
    was to present an easy-to-use interface for researchers to easily perform
    complex tasks such as cleaning the data, filtering, applying complex signal
    processing methods and visualizing data. It featured development of custom
    visualizations and interfaces for spectral analysis (PSD and spectrograms)
    and source-separation (ICA).\\
    $\circ$ Contributing to a wide open-source project for neurophysiological
    data processing.\\
    $\circ$ Working alongside researchers to identify needs and bottlenecks
    in the user experience of the product.The software is now being used by
    researchers at Campus Biotech to process their data.\\
\textbf{Utilized:} Python (PyQT, Matplotlib for interactive graphs, Numpy
MNE), advanced signal processing
\\[0.4cm]
%------------------------------------------------------------------------------
\textbf{FOOTBALL TOURNAMENT PREDICTION}\\
    Modelling a football tournament using Poisson regression models by giving
    teams offensive and defensive attributes, and simulating outcomes to
    predict the winner. At the end of the project, we were able to give a
    probability of winning for each team with a confidence interval. The
    project involved the transformation of the raw data and the conception of
    a model.\\
\textbf{Utilized:} R, Advanced Regression Models, Monte-Carlo Simulations
\\[0.4cm]
%------------------------------------------------------------------------------
\textbf{ROOM HEATING OPTIMIZATION}\\
    The goal of this project over the span of a year, was to find an optimal
    polygon shaped room (square, pentagon etc.), in order to maximize the
    average temperature.\\
    $\circ$ Modelling the situation with heat equation, using adaptive meshes\\
    $\circ$ Developing optimization algorithms from scratch\\
\textbf{Utilized:} Python, Matlab, Matlab to Python pipeline
\\[0.4cm]
%------------------------------------------------------------------------------
\textbf{AURORA FORECAST VISUALIZATION}\\
\url{https://northern-lights.herokuapp.com/}\\
    Personal project involving the development of a website for that displays
    the northern lights forecast on an interactive map. The application 
    extracts, transform and displays external data from the U.S SPWC.\\
\textbf{Utilized:} Python (Flask), d3.js
\\[0.4cm]
%------------------------------------------------------------------------------
\section{REFERENCES}
    \textbf{Gwénaël BIROT} \\
    EEG/BCI Platform Manager, \\
    Fondation Campus Biotech Geneva,\\
    \url{gwenael.birot@fcbg.ch}

\end{document}
